{\documentclass [10pt,a4paper]{book}

\begin{document}

\begin{flushright}
	

COLLABORATIVE E-RESEARCH      \textbf{81}
	
\end{flushright}		 
However, the creator of the community has the capacity to push a notice via email to new members, inviting them to join the community. CommunityZero provides a group calendar for each community, to which members can add or delete appoint-ments. Like SharePoint, it provides a polling service to gather and summarize opinions from community members. Currently, basic CommunityZero core services are pro-vided free of charge, but customized services with subscriber branding, enhanced secu-rity, and the capacity to host many communities are available on a fee-for-service basis.

\begin{flushleft} 
\textbf{GROOVE NETWORKS: PEER-TO-PEER COLLABORATION SOFTWARE}
\end{flushleft} 
The final software suite we review offers a more radical means of providing collabora-tive services, one which Internet gurus have speculated will change the way we cur-rently collaborate using the Net. The architecture of services originally provided on the Net and exemplified by both SharePoint and CommunityZero follow a model developed in telephone communication. In this model, all intelligence (e.g., billing information switching, directory services) is located at the center of the network, allowing for unintelligent devices (i.e., telephone hand sets) to connect to these ser-vices. Point-to-point technology takes an opposite approach and relies on the com-puting power of devices operating at the edge of the network to provide intelligent services, thus reducing reliance on central services. In the familiar Napster example, a central service merely tracks nines from the disk drives of logged-on members—all transfer and sharing happens directly between these members. Other peer-to-peer ser-vices such as Aimster and Gnutella eliminate this central database service altogether, as members broadcast requests for specific files to their neighbors and transfer these files to member computers. 


A company called Groove Network first applied this peer-to-peer concept to col-laborative work. Their collaborative system, Groove software, works by downloading a suite of software to connect directly to other users, share files, and communicate without going through a central Web server. Shared files are stored on the local users' hard drive, and communication services, including multi-point text and audio chat, are supported by broadcasting communication directly to the other members of the research team from each machine. This approach will be attractive to those e-researchers who are not in a position to manage or to add programming to a Web server, and who are not comfortable with sharing or storing sensitive data on a com-mercial site. Groove also supports asynchronous threaded discussion lists and inte-grates with the users' default mailer for private email conversations. Video conferencing can be supported through integration with Microsoft's NetMeeting tech-nology. Groove addresses the need for document control by insuring that two users do not edit the same document simultaneously. When documents are retrieved from the shared file space, any subsequent requests for editing of the document are not allowed. Instead, Groove creates a second copy with a slightly different name. This alerts the second user to its "in use" status, but still allows for editing, even though the two doc-uments must be compared and merged before a single document is created. This com-parison can be automated to some degree by using the "compare documents" feature 
\end{document}
