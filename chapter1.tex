% !TEX TS-program = XeLaTeX
% !TeX root=main.tex
\chapter{}
\begin{latin}
resides on a Windows server and provides a list of services, as illustrated in Figure 6.1. One of the unique features of the SharePoint service is the capacity for members of the team to alter the Web site directly from their Web browser, without having to craeate or upload docauments or HTML. pages. The communication services include threaded discussion groups that are presented in a format similar to typical newsgroups or computer conferences. Besides sharing documents, SharePoint provides customizable notification services using email to push notification to subseribers when information changes in either discussion groups or document libraries. 
SharePoint aslo has a polling feature that  allows members to query each other and instantly display the results. As seen in the screen capture in Figure 6.1, SharePoint is integrated with other Microsoft Office products, and thus will be most useful for teams that already use this suite of products. For example, SharePoint itself does not provide a  calendar feature, but can export items directly to the services built into Microsoft Outlook's calendar applications. SharePoint services must be installed on an internal company or institutional Web server and can  be purchased from external Web service providers. As the services and built into the cost of  the full-featured versions of Office XP, the cost for installation and operation is quite nominal; however, installation on a central server normally requires the cooperation of the information rechnology services of a central organization.
\begin{figure}[H]
\centering
 \includegraphics*[scale=1]{1}
\caption{ Screen Shot from Microsoft SharePoint. ${}^{TM}$ Reprinted by permission from Microsoft Corporation.}
\end{figure}
%%%%%%%%%%%%%79
\section*{COMMUNTYZERO}
CommunityZero supplies centralized collaborative services that give ndividual users access to any  number of  '' communities '' -- some of which could be the focus of an e-research project. Unlike SharePoint, all CommunityZero communities are  located on a single, central Web server. Individuals enroll  for  free and are allowed to participate in as  many different communities as  they  wish to create  or join. CommunityZero provides a larger suite of services to members than SharePoint and  serves as a customizable portal for team members. Figure 6.2 provides a screen  shot of a CommunityZero site that promotes the services provided.

CommunityZero provides asynchronous, threaded discussions; a << contributions >> area where members can provide annotated links to sites or resources on the  Web; a notice board feature for short announcements; and a list feature for creating itemized lists such as << cheklists>> and << to do >> lisys. CommunityZero provides  real-time chat rooms and a series of << newsfeeds>> from press agencies and other news organizations for team members. Like SharePoint, CommunityZero provides space for uploading and retrieving documents created by team members, but does not provide the notification services available in SharePoint that alert members when items change.
\begin{figure}[H]
\centering
 \includegraphics*[scale=1]{2}
\caption{ Screen Shot from CommunityZero. ${}^{TM}$}
\end{figure}
%%%%%%191
However, the creator of the community has the capacity to push a notice via email to new members, inviting them to join the community, CommunityZero provides a group calendar for each  community, to which members can add or delete appointments. Like SharePoint, it provides a polling service to gather and summarize opinions from community members. Currently, basic CommunityZero core services are provided free of charge, but customized services with subseriber branding, enhanced security, and the capacity to host many communities are available on a fee-for-service basic.
\section*{GROOVE NETWORKS; PEER-TO-PEER COLLABORATION SOFTWARE}
The final software suite we review offers a more radical means of providing collaborative services, one which Internet gurus have speculated will change the way we currently collaborate using the Net. The architecture of services originally provided on the Net and exemplified by both SharePoint and CommunityZero follow a model developed in telephone communication. In this model, all intelligence (e.g., billing information switching, directory services) is located at the  center of the network, allowing for unintelligent devices (i.e., telephone hand sets) to connect to these services. Point-to-point technology takes  an opposite approach and relies on the computing power of devices operating at the edge of the network to provide intelligent services, thus reducing reliance on central servies. In the familiar Napster example, a central service merely tracks tunes from the disk drives of logged-on memberes--all transfer and sharing happens directly between these members. Other peer-to-peer services such as Aimster and Gnutella eliminate this central database service altogether, as members broadcast requests for specific files to their neighhors and transfer these files to member computers.

A company called Croove Network first applied this peer-to-peer concept to collaborative work. Their collaborative system, Groove software, works by downloading a suite of software to connect directly to other user, share files, and communicate without goingthrough a central Web server. Shared files are stored on the local users' hard drive, and communication services, including multi-point text and audio chat, are supported by broadcasting communication directly to the other members of the research team from each machine. This approach will be attractive to those ereseachers who are not in a position on manage or to add programming to a Web server, and who are not comfortable with sharing or stroing sensitive data on a commercial site. Groove also supports asynchronous thereaded discussion lists and integrates with the users' defaulr mailer for private email conversations. Video conferencing can  be supported  through integration with Microsoft's NetMeeting technology. Groove addresses the need for documentcontrol by insuring that two users do not edit the same document  simultaneously. When documents are retrieved from the shared file space, any subsequent requests for  editing of the document are not allowed. Instead, Groove creates a second copy with a slightly different name. This alerts the second user to its < in use >  status, but still allowsfor editing, even though the two documents must be compared  and merged before a single document is created. This comparison can be automated to some degree by using the << compare documents >> feature
%%%%















\end{latin}












